\chapter{Abstract}
\label{chap:abstract}

Collecting experimental data is of considerable importance in the study of cavitation. Indeed, empirical methods are often used in the design of ship propellers.
In this scenario, acquiring a deeper and more detailed knowledge of cavitation phenomena would allow a more comprehensive understanding of their effects and risks and, therefore, a more efficient design.

This thesis presents an approach based on computer vision techniques for studying cavitation. 
Specifically, the research focuses on analyzing the dynamics of cavitating structures and their influence on cavitation-related effects, such as underwater radiated noise and cavitation erosion. To this end, two distinct case studies were considered.
For each case, the proposed approach demonstrated that applying computer vision to the study of cavitation provides valuable data that are otherwise challenging or impossible to obtain. Furthermore, combining the information derived from computer vision algorithms with the time resolution offered by high-speed video recordings allows for quantitative observation of the dynamics of cavitation phenomena.

This kind of data provides valuable information, whether for estimating a phenomenon's erosive potential or predicting the produced acoustic radiation. 
Unfortunately, using these techniques in conventional tests conducted in cavitation tunnels is not straightforward due to various environmental factors that significantly complicate measurements.
For this reason, the thesis details the method's development, showing the challenges encountered and the solutions adopted at each stage.
The resulting method proved robust and adaptable to different scenarios with a moderate level of effort. It enabled the collection of detailed measurements of cavitation and its dynamics, paving the way for new and insightful studies.



