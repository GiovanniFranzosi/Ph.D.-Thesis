\chapter{Conclusions}
\label{chap:chapter7}

This thesis presents a computer vision-based approach for measuring cavitation. This approach was applied to two case studies to investigate the relationship between cavitation dynamics and related phenomena. 
Although both studies are preliminary, the results are highly promising. The employed techniques have demonstrated the capability to accurately measure cavitation and its evolution over time.

The first case study aimed to investigate erosive cavitation by analyzing bubble cavitation dynamics.
Computer vision enabled the measurement of bubble sizes and the tracking of their dynamics over time.
By combining these data with a model describing bubble dynamic, key parameters closely related to erosive cavitation, i.e., position, duration, and pressure impulse magnitude, have been derived.

The second case study investigated the relationship between tip vortex cavitation dynamics and the underwater noise it generates. In this case, computer vision techniques were used to assess the large-scale dynamics of the cavitating vortex volume and radius.
These measurements were compared with flow observations at the propeller tip and noise measurements. 
Analyzing these data enabled to examine the correlation between vortex dynamics and the macroscopic features of the radiated noise spectrum.

Acquiring such information is generally challenging, if not impossible, using conventional measurement techniques.

Despite these encouraging aspects, it is important not to neglect the challenges associated with this technique. Applying computer vision to the cavitation tunnel environment presents several complexities. First and foremost, analyzing cavitation dynamics requires high-quality hardware, such as high-speed cameras and an appropriate illumination and synchronization system. Additionally, it necessitates significant optical access to the test section of the tunnel.
To properly position the cameras, the prism and the lights required for measurement, the optical access to the test section must be wide and easily accessible.
Finally, the technique's limitations must be considered. Although the proposed measurement system has proven to be quite robust, it is important to highlight that the effectiveness of these techniques is highly dependent on the quality of the acquired images. For this reason, it is often necessary to calibrate specific parameters or modify certain portions of the algorithm to tailor the procedure to each case study.

The author believes that these limitations should not preclude the use of computer vision in the study of cavitation. Indeed, when appropriately implemented, these techniques can deepen our understanding of cavitation. Computer vision offers the possibility of quantitatively measuring the size of cavitation phenomena and tracking their dynamics over time. Combined with more traditional measurement methods, this approach could contribute to a more comprehensive understanding of cavitation and enable the development of more accurate semi-empirical models.


